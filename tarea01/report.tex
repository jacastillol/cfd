%\documentclass[9pt,technote,twoside,letterpaper,twocolumn]{IEEEtran}
\documentclass[9pt,technote,twoside,letterpaper,onecolumn]{IEEEtran}

\usepackage{amsmath}
\usepackage{amssymb}
\usepackage[utf8]{inputenc}
\usepackage{graphicx}
\usepackage{xcolor}
\usepackage{multicol} % figure}[!hbt] by figure*}[!hbt]
\usepackage{float}
\usepackage{listings}
\lstset{language=[90]Fortran,
  numbers=left,
  numberstyle=\tiny,%frame=double,framexleftmargin=15pt,
  xleftmargin=2em,
  basicstyle=\scriptsize\ttfamily,
  keywordstyle=\color{blue},
  keywordstyle=[2]{\color{red!70!black}},
  commentstyle=\color{green!70!black},
  morecomment=[l]{!\ },% Comment only with space after !
  otherkeywords={;,::,&},
  morekeywords=[2]{&},
}

\title{Introducción a las Diferencias Finitas aplicadas a CFD}
\author{Jaime~Andrés~Castillo-León}

\begin{document}

\begin{multicols}{2}
\maketitle
%\begin{abstract}
%  Bla, bla....
%\end{abstract}

%\begin{IEEEkeywords}
%  Bla, bla....
%\end{IEEEkeywords}

%\section{Introducción}
%\label{sec:intro}

\section{Exactitud de esquemas de diferencias finitas}
\label{sec:exac}
En el siguiente problema, se desea aproximar la primera y segunda derivada de una función analitica $f(x)$,  definida como:
\begin{equation}
  f(x)\,=\,\exp(\sin(x))\,+\,0.5\cos(\kappa_1x)\,-\,0.8\sin(\kappa_2x)
  \label{eq:fun}
\end{equation}
en donde $x\in[-\pi,\pi]$, $\kappa_1 = \frac{\pi}{2\Delta x_{max}}$, $\kappa_2 = \frac{3\kappa_1}{4}$ y $\Delta x_{max}=0.1$ corresponde al tamaño de malla más pequeño de este experimento.

La primera derivada, de forma analítica es:
\begin{align}
  \frac{df}{dx}\,&=\,\cos(x)\exp(\sin(x))\notag\\
&\quad-\,0.5\kappa_1\sin(\kappa_1x)\,-\,0.8\kappa_2\cos(\kappa_2x)
  \label{eq:dfun}
\end{align}
y la segunda derivada es:
\begin{align}
  \frac{d^2f}{dx^2}\,&=\,(\cos^2(x)-\,\sin(x))\exp(\sin(x))\notag\\
&\quad-\,0.5\kappa^2_1\cos(\kappa_1x)\,+\,0.8\kappa^2_2\sin(\kappa_2x)
  \label{eq:d2fun}
\end{align}

En la figura Fig.\ref{fig:funx}, se puede apreciar como la función $f(x)$, presenta altos valores en la primera y la segunda derivada. Esta característica será analizada más adelante como una influencia importante en el error del estencil.
\begin{figure}[H]
  \centering
  \includegraphics[width=9.0cm]{images/001Funcion.pdf}\\
  \label{fig:funx}
  \caption{Funcion de prueba para aproximar las primeras dos derivadas utilizando esquemas de diferencias finitas: (arriba) $f(x;\kappa_1,\kappa_2,\Delta_x)$, (centro) $\frac{df}{dx}$, (abajo) $\frac{d^2f}{dx^2}$}
\end{figure}

\subsection{Aproximación de las derivadas de $f(x)$}
\label{sec:aproxDFM}

% TODO: Enumerate
A continuación se presenta la deducción general de un esquema de diferencias finitas. En primer lugar se expresa la función $f(x)$ como una serie de Taylor alrededor de punto $x_i$, como $f(x)\,=\,\sum^\infty_{n=0}\,\frac{(x-x_i)^n}{n!}f^{(n)}(x_i)$. 

De forma similar se puede expresar la derivada de $f(x)$, es decir $\frac{df}{dx}$, como $f^{(1)}(x)\,=\,\sum^\infty_{n=0}\,\frac{(x-xi)^n}{n!}f^{(n+1)}(x_i)$.

Suponiendo que se tiene una malla uniforme, $\Delta\,=\,(x_{i+1}-x_i)$, se puede aproximar $f_{i+k}\,=\,f(x_{i+k})\,=\,f(k\Delta+x_i)$, en un punto $x_{i+k}$ vecino a $x_i$, como
\begin{equation}
  f_{i+k}\,=\,\sum^\infty_{n=0}\,\frac{(k\Delta)^n}{n!}f^{(n)}_i
  \label{eq:xi+k}
\end{equation}

Para diseñar un esquema de la primera derivada $f^{(1)}(x_i)$ de tamaño $m-l$, es decir $f^{(1)}_i\,=\,a_lf_{i+l}+a_{l+1}f_{i+l+1}+\cdots+a_mf_{i+m}$, en donde $m,l\in\mathbb{Z}$, se plantea que la suma de series de Taylor de los puntos involucrados en el esquema deseado:
\begin{equation}
  \sum^m_{k=l}\,a_kf_{i+k}\,=\,\sum^m_{k=l}\sum^\infty_{n=0}\,a_k\frac{(k\Delta)^n}{n!}f^{(n)}_i
  \label{eq:sysScheme}
\end{equation}
truncando la serie de Taylor ec.(\ref{eq:sysScheme}), es posible plantear el siguiente conjunto de ecuaciones, una ecuación para cada derivada de orden $n$ de la función $f(x)$:
\begin{equation}
  f^{(n)}_{i}:\qquad \sum^m_{k=l}\,a_k\frac{(k\Delta)^n}{n!}\,=\,c_n
  \label{eq:sysSet}
\end{equation}
en donde $n$ es un conjunto de $m+l$ numeros naturales que incluyen al cero. 

Los valores de $c_n$ se tomarán como únicamente dos posibles valores $\{0.0,1.0\}$ y significan: primero, si $c_n=0$ es como si el error de la serie de Taylor del esquema deseado en la derivada $f^{(n)}$ es cero y segundo, si $c_n=1.0$ entonces el error de orden $n$ no es nulo. Para obtener la derivada de orden $1$, se debe poner $c_1=1.0$.  

\subsection{Diseño del Stencil}
\label{sec:design}

Es posible reescribir la ecuación ec.(\ref{eq:sysSet}), como un sistema matricial de la siguiente forma
\begin{equation}
  \left[\frac{(k\Delta)^n}{n!}\right]_{nk}\cdot\left(a_k\right)\,=\,\left(c_n\right)
  \label{eq:designStencil}
\end{equation}

En el caso de diseños compactos es posible llegar al siguiente sistema:
\begin{equation}
  \left[\frac{(k\Delta)^n}{n!}\,,\,\frac{(p\Delta)^{n-1}}{n!}\right]_{nk}\cdot\left(
  \begin{array}{c}
a_k\\
b_p
  \end{array}
\right)\,=\,\left(c_n\right)
  \label{eq:designCompactStencil}
\end{equation}

\subsubsection{Diferencias Centradas de $4^{to}$ Orden}
\label{sec:cds4}

El estencil plantea el siguiente sistema, según la ec.(\ref{eq:designStencil}):
{\small
\begin{equation}
  \left[\begin{matrix}
      1 & 1 & 1 & 1 & 1\\
      -2 & -1 & 0 & 1 & 2\\
      2 & \frac{1}{2} & 0 & \frac{1}{2} & 2\\
      - \frac{4}{3} & - \frac{1}{6} & 0 & \frac{1}{6} & \frac{4}{3}\\
      \frac{2}{3} & \frac{1}{24} & 0 & \frac{1}{24} & \frac{2}{3}
    \end{matrix}\right]
  \cdot \mathbf{a}\,=\,
  \left[\begin{matrix}
      0\\
      1\\
      0\\
      0\\
      0
    \end{matrix}\right]
  \label{eq:cds4}
\end{equation}
}
se obtiene 
\begin{equation}
  \Delta f^{(1)}_i\,=\,\frac{1}{12}f_{i-2}-\frac{2}{3}f_{i-1}+\frac{2}{3}f_{i+1}-\frac{1}{12}f_{i+2}
  \label{eq:cds4stencil}
\end{equation}
de forma similar se puede obtener para las fronteras estenciles hacia adelante de orden 4:
{\small
\begin{equation}
  \left[\begin{matrix}
      1 & 1 & 1 & 1 & 1\\
      -1 & 0 & 1 & 2 & 3\\
      \frac{1}{2} & 0 & \frac{1}{2} & 2 & \frac{9}{2}\\
      - \frac{1}{6} & 0 & \frac{1}{6} & \frac{4}{3} & \frac{9}{2}\\
      \frac{1}{24} & 0 & \frac{1}{24} & \frac{2}{3} & \frac{27}{8}
    \end{matrix}\right]  
  \cdot \mathbf{a}\,=\,
  \left[\begin{matrix}
      0\\
      1\\
      0\\
      0\\
      0
    \end{matrix}\right]
  \label{eq:cds4-bound}
\end{equation}}
se obtiene 
\begin{equation} % [-5/6], [3/2], [-1/4], [-1/2], [1/12] % 0,1,2,3,5
  \Delta f^{(1)}_i\,=\,-\frac{1}{4}f_{i-1}-\frac{5}{6}f_{i}-\frac{3}{2}f_{i+1}-\frac{1}{2}f_{i+2}+\frac{1}{12}f_{i+3}
  \label{eq:cds4stencilbound}
\end{equation}

Para implementar unas diferencias finitas de cuarto orden, incluidas las fronteras se tiene:
\begin{align} % [-25/12], [4], [-3], [4/3], [-1/4]
  \Delta f^{(1)}_i\,&=\,-\frac{25}{12}f_{i}+4f_{i+1}-3f_{i+2}+\frac{4}{3}f_{i+3}-\frac{1}{4}f_{i+4}\notag\\
  \Delta f^{(1)}_i\,&=\,-\frac{1}{4}f_{i-1}-\frac{5}{6}f_{i}+\frac{3}{2}f_{i+1}-\frac{1}{2}f_{i+2}+\frac{1}{12}f_{i+3}\notag\\
  \Delta f^{(1)}_i\,&=\,\frac{1}{12}f_{i-2}-\frac{2}{3}f_{i-1}+\frac{2}{3}f_{i+1}-\frac{1}{12}f_{i+2}\\
  \Delta f^{(1)}_i\,&=\,-\frac{1}{12}f_{i-3}+\frac{1}{2}f_{i-2}-\frac{3}{2}f_{i-1}+\frac{5}{6}f_{i}+\frac{1}{4}f_{i+1}\notag\\
  \Delta f^{(1)}_i\,&=\,+\frac{1}{4}f_{i-4}-\frac{4}{3}f_{i-3}-4f_{i-1}-3f_{i-2}+\frac{25}{12}f_{i}\notag
\end{align}

\subsubsection{Diferencias Centradas de $2^{do}$ Orden}
\label{sec:cds2}
\begin{align} % [-25/12], [4], [-3], [4/3], [-1/4]
  \Delta f^{(1)}_i\,&=\,-\frac{3}{2}f_{i}+2f_{i+1}-\frac{1}{2}f_{i+2}\notag\\
  \Delta f^{(1)}_i\,&=\,-\frac{1}{2}f_{i-1}+\frac{1}{2}f_{i+1}\\
  \Delta f^{(1)}_i\,&=\,+\frac{1}{2}f_{i-2}-2f_{i-1}+\frac{3}{2}f_{i}\notag
\end{align}

\subsubsection{Diferencias Centradas de $6^{to}$ Orden}
\label{sec:cds6}

\begin{center}\scriptsize
\begin{tabular}{ |c|c|c|c|c|c|c|c| } 
  \hline
  k & \scalebox{.8}{$f_{i-3+k}$} & \scalebox{.8}{$f_{i-2+k}$} & \scalebox{.8}{$f_{i-1+k}$} & \scalebox{.8}{$f_{i+k}$} & \scalebox{.8}{$f_{i+1+k}$} & \scalebox{.8}{$f_{i+2+k}$} & \scalebox{.8}{$f_{i+3+k}$} \\
  \hline 
  -3&-49/20&6&-15/2&20/3&-15/4&6/5&-1/60\\ % [-49/20], [6], [-15/2], [20/3], [-15/4], [6/5], [-1/6] 
  -2&-1/6&-77/60&5/2&-5/3&5/6&-1/4&1/30\\ % [-77/60], [5/2], [-1/6], [-5/3], [5/6], [-1/4], [1/30]
  -1&1/30&-2/5&-7/12&4/3&-1/2&2/15&-1/60\\ % [-7/12], [4/3], [-2/5], [-1/2], [1/30], [2/15], [-1/60]
  0&-1/60&3/20&-3/4&0&3/4&-3/20&1/60 \\ % [0], [3/4], [-3/4], [-3/20], [3/20], [1/60], [-1/60]
  1&1/60&-2/15&1/2&-4/3&7/12&2/5&-1/30 \\
  2&-1/30&1/4&-5/6&5/3&-5/2&77/60&1/6\\
  3&1/60&-6/5&15/6&-20/3&15/2&-6&49/20\\
  \hline
\end{tabular}
\end{center}

\subsection{Acerca del error de truncamiento de los esquemas}
\label{sec:error}
A continuación se muestra el orden de exactitud o el error de truncamiento ocasionado por los esquemas de segundo, cuarto y sexto orden estudiados en el documento. Como se puede apreciar en la Figura.\ref{fig:errtrunc}, cuando el tamaño de la malla se reduce, efectivamente se reduce el error, pero antes debe pasar una zona, la cual no reduce el error. Esto es debido a que en esa zona la derivada de la función es mucho mayor que el el efecto de $\Delta$. 

\begin{figure}[H]
  \centering
  \includegraphics[width=8.0cm]{images/002Errores_df.pdf}
  \caption{Efecto del tamaño de la malla para $f(x)$ con $\Delta x_{max}=0.1$ y las derivadas de la función en $\frac{df}{dx}$ (Observar recuadro de Zoom).}
  \label{fig:errtrunc}
\end{figure}

Se puede afirmar que para $\Delta<4*10^{-3}$, el error se reduce sin importar el tiempo de esquema utilizado. Para valores de malla mayores a este valor, las mallas arrojaran valores errados. Es importante aclara que este tipo de comportamiento es exclusivo de la función de prueba $f(x)$, que como se dijo anteriormente tiene derivadas con valores elevados.
 
Como es de esperarse, al reutilizar el esquema para obtener la segunda derivada, el cálculo de la segunda derivada acumula los errores de la primera derivada (Vease la Figura.\ref{fig:err2trunc}). Lo que requiere para una aproximación aceptable que $\Delta_{cds2}<10^{-4}$. Se puede notar que la velocidad con que cae el error de la segunda derivada es mucho menor que la velocidada con que cae el error de la primera derivada, en por lo menos dos ordenes de magnitud.

\begin{figure}[H]
  \centering
  \includegraphics[width=8.0cm]{images/003Errores_d2f.pdf}
  \caption{Efecto del tamaño de la malla $\Delta$ y las derivadas de la función en $\frac{d^2f}{dx^2}$ (Observar recuadro de Zoom).}
  \label{fig:err2trunc}
\end{figure}

Para comporbar que el efecto de la derivada es intrínseco de la función $f(x)$, la función de la ecuación ec.(\ref{eq:fun}), se ha modificado utilizando un $\Delta x_{max}\,=\,1.0$, con lo cual las derivadas se reducen considerablemente. Al aplicar los esquemas de diferenciación a dicha funcion, se puede observar que cualquier $\Delta<0.1$ (ver Figura.\ref{fig:errdxOthertrunc}), disminuye el error, lo cual no ocurre para la funcion $f(x)$ original (ver Figura.\ref{fig:errtrunc}).

En la Figura.\ref{fig:errdxOthertrunc}, aparace un codo para la aproximación de sexto orden, en donde el error en lugar de disminuir comienza a crecer. Este efecto es debido al \emph{error de redondeo} que se debe a la precisión de la máquina y la forma como aproxima los números reales.

\begin{figure}[H]
  \centering
  \includegraphics[width=8.0cm]{images/004Errores_df_10dx.pdf}
  \caption{Efecto de las derivadas de la función $f(x)$ con $\Delta x_{max}=1.0$ en $\frac{df}{dx}$.}
  \label{fig:errdxOthertrunc}
\end{figure}

\subsubsection{Error CDS 2, 4 y 6}
\label{sec:errcds4}
Lo explicado anteriormente se puede ver en las siguientes ecuaciones de orden de exactitud, en donde se evidencia el efecto del tamaño de malla y el valor particular de la funcion $f(x)$:
\begin{align}
  \epsilon_{cds2}(x)\,&=\,k_2\Delta_i^2\left.\frac{d^3f}{dx^3}\right|_{x_i}
  \label{eq:errcds2}\\
  \epsilon_{cds4}(x)\,&=\,k_4\Delta_i^4\left.\frac{d^5f}{dx^5}\right|_{x_i}
  \label{eq:errcds4}\\
  \epsilon_{cds6}(x)\,&=\,k_6\Delta_i^6\left.\frac{d^7f}{dx^7}\right|_{x_i}
  \label{eq:errcds6}
\end{align}
donde $k_i$ es una constante, que se obtiene de aumentar una fila en la ec.(\ref{eq:designStencil}).

\subsection{Implementación Diseño Stencils}
\label{sec:imple}

La implementación Listing.\ref{lst:label_stencil} muestra una función en Fortran 90, que crea cualquier esquema de Diferencias Finitas, propuesta por ec.(\ref{eq:designStencil}):
\begin{center}
\begin{lstlisting}[caption={Diseño de Stencils},label={lst:label_stencil}]
! Crea el estencil de un esquema no-compacto
subroutine get_Kernel(Kernel, ord, shift, dx)
  implicit none
  integer, intent(in) :: ord, shift
  real*8, intent(in) :: dx
  real*8, dimension(:), intent(out) :: Kernel
  integer :: j, delta, info
  integer, dimension(ord) :: pivs
  real*8, dimension(ord,ord) :: A
  real*8, dimension(ord) :: b
  ! Ensamble del sistema
  A = 0.0D0
  b = 0.0D0; b(2) = 1.0D0
  do delta=shift,order+shift-1
    do j=0,order-1
      A(j+1,delta-shift+1) = &
          (real(delta)**j/factorial(j))*dx
    end do
  end do
  ! Solucion
  call dgesv(order, 1, A, ord, pivs, &
             b, ord, info)
  Kernel = b
end subroutine get_Kernel
\end{lstlisting}
\end{center}

\subsection{Acerca del error de redondeo}
\label{sec:errorred}

Continuando con el efecto del codo de la Figura.\ref{fig:errdxOthertrunc}, se implementó los algoritmo para que se pudiera explorar los errores de redondeo presentes en los tipos de dato de \emph{punto flotante} implementados en Fortran, mediante el estandar \emph{IEEE-754}. Esto son \verb.single., \verb.double. y \verb.quad. representados por la siguiente Listing.\ref{lst:types}. 
\begin{lstlisting}[caption={Punto flotante},label={lst:types}]
module parametros
  implicit none
  integer, parameter :: sp = & !float32
      selected_real_kind(6, 37)
  integer, parameter :: dp = & !float64
      selected_real_kind(15, 307)
  integer, parameter :: qp = & !float128
      selected_real_kind(33, 4931)
  integer, parameter :: Long = &
      selected_real_kind(15, 307)
end subroutine parametros
\end{lstlisting}

Cada tipo de dato maneja un nivel de redondeo de acuerdo a la cantidad de memoria y a la forma en que almacena las cantidades. Un número representado por punto flotante, es una aproximación finita de los números reales, no es posible representar la continuidad de la recta de los reales, asi que la distancia más pequeña entre dos numeros pequeños representa la precision del sistema flotante. Por ejemplo, la definición del epsilon de la máquina $\epsilon_{maq}$ es el número más pequeño que puede representar el numero flotante mayor a la unidad.

\begin{figure}[H]
  \centering
  \includegraphics[width=9.0cm]{images/005Regions_dx1.pdf}
  \caption{Errores de truncamiento y redondeo de la función $f(x)$ con $\Delta x_{max}=1.0$ en $\frac{df}{dx}$ según el timpo de dato y el tamaño de malla.}
  \label{fig:regions_dx10}
\end{figure}

\begin{figure}[H]
  \centering
  \includegraphics[width=9.0cm]{images/005Regions_dx01.pdf}
  \caption{Efecto de las derivadas de la función $f(x)$ con $\Delta x_{max}=0.1$ en $\frac{df}{dx}$.}
  \label{fig:regions_dx01}
\end{figure}

\begin{figure}[H]
  \centering
  \includegraphics[width=9.0cm]{images/006Regions_df2_dx01.pdf}
  \caption{Efecto de las derivadas de la función $f(x)$ con $\Delta x_{max}=0.1$ en $\frac{d^2f}{dx^2}$.}
  \label{fig:regions_d2fdx01}
\end{figure}

\subsection{Implementación de la primera y la segunda derivada}
\label{sec:imp1rader}
Es posible encontrar la primera y segunda derivadas de $f(x)$, mediante:
\begin{equation}
  f'(x)\,=\,\mathbf{D}\cdot f(x)
  \label{eq:matdiff}  
\end{equation}
\begin{equation}
  f''(x)\,=\,\mathbf{D}^2\cdot f(x)\,=\,\mathbf{D}\cdot f'(x)
  \label{eq:matdiff2}  
\end{equation}
en donde $f$ y $x\in\mathbb{R}^{n}$, $n$ el número de nodos de la malla y $\mathbf{D}\in\mathbb{R}^{n\times n}$ es una matriz banda que se define por los esquemas de diferenicas finitas que se utilizan. Sin embargo la implementación de dicha ecuación requiere de ciertas bondades para poder evaluar el error de redondeo de la Figura.\ref{fig:regions_dx10}.  
\subsubsection{Matriz Densa de diferenciación}
\label{sec:imp_matrix}
Implmenentar $\mathbf{D}$ como una matriz densa es la forma más sencilla, pero para $n=10^4$, se requiere 6.4GB de la memoria RAM utilizando precisión de \verb.double.. Lo cual comvierte este método en una forma inviable de analizar el error de redondeo, ya que la memoria del programa se llenaría antes de que se logre el error de redondeo. 
\subsubsection{Matriz Banda basada en el estencil}
\label{sec:imp_matrix_band}
La matriz densa se puede cambiar por dos submatrices que representan los kernels en el contorno y el dominio. De esta forma el exceso de memoria de la matriz $\mathbf{D}$ queda resuelto, al tamaño de los estenciles. Sin embargo para encontrar la segunda derivada se requiere de $\mathbf{D}^2$ o de $f'(x)$, lo cual representa un requerimiento en memoria de 640MB para un arreglo de $f'\,\in\mathbb{R}^7$, lo cual convierte el problema en un problema en memoria nuevamente.
\subsubsection{Convolución de la malla usando el tamaño del estencil}
\label{sec:imp_conv}
Por último en lugar de almacenar $f^{'}(x)$ en un arreglo total, se puede almacenar $f^{'}_{\Omega}$ solo en al vecindad del estencil para obtener la segunda derivada. De esta forma se puede calcular cualquier tamaño de malla, con la limitante de que ahora el problema es el tiempo de ejecución. En el caso de la malla de $5*10^8$, el tiempo de ejecución con un Core i7, en un solo núcleo fue de 380 minutos para una precisión de \verb.quad.. Como ventaja es posible correr el mismo programa siete veces uno por cada núcleo, dejando el tiempo total de computo en 4 horas y 20 minutos (es decir 380 minutos). El desempeño en tiempo se puede ver en la Figura.\ref{fig:performance}, en donde se puede calcular la complejidad exponencial en tiempo, como $\mathcal{O}(b\,n^a\,)$ y $N$ es el tamaño de la malla.

\begin{figure}[H]
  \includegraphics[width=9.0cm]{images/007Performance.pdf}
  \caption{Efecto de las derivadas de la función $f(x)$ con $\Delta x_{max}=0.1$ en $\frac{df}{dx}$.}
  \label{fig:performance}
  \centering
\end{figure}

Es posible calcular $a$ y $b$ concluyendo que $a<1$, para cualquiera de los casos de punto flotante.
\begin{center}\scriptsize
\begin{tabular}{ |c|c|c|c| } 
  \hline
  Params & valor & precision & std. err \\
  \hline 
  $a_s$& 0.935237&+/- 0.02513&2.687\%\\
  $b_s$& 3.7883e-08&+/- 1.683e-08&44.43\%\\  \hline
  $a_d$& 0.935706&+/- 0.01297&1.386\%\\
  $b_d$& 7.25366e-08&+/- 1.881e-08&25.93\%\\  \hline
  $a_q$& 0.840048&+/- 0.01265&1.506\%\\
  $b_q$&1.8651e-05&+/- 4.701e-06&25.2\%\\
  \hline
\end{tabular}
\end{center}

Por ultimo la impementación de este caso, lleva a los siguientes pasos:
\begin{align}
  f^{'}_i\,&=\,K_r\cdot f_{\Omega_i}\label{eq:buffdiff}\\
  f^{''}_i\,&=\,\sum_{k,j\in\Omega_{ij}}D_{ik}D_{kj} f_j\label{eq:optimDiff}
\end{align}
en donde $f_{\Omega_i}$ depende de los indices $\Omega_i$ del kernel $K_r\in\{K_b, K_d\}$, dependiendo de la ubicación del nodo sera $K_b$ para el esquma cercano al contorno o $K_d$ para nodos internos del dominio, ver detalles de la implementación en los Anexos de este documento, Sección.\ref{sec:opt_alg}.

\section{Ecuación de difusión-convección 1D}
\label{sec:dif-conv}
A continuación se analiza la ecuación de difusión-convección en estado estable para el caso unidimensional:
\begin{equation}
  \frac{\partial\,\rho u \phi}{\partial x}\,=\,\frac{\partial}{\partial x}\,\left(\Gamma\frac{\partial\,\phi}{\partial x}\right)
  \label{eq:dif-conv}
\end{equation}
sujeto a las condiciones de frontera,
\begin{align}
  \phi(x=0)\,&=\,\phi_o\\
  \phi(x=L)\,&=\,\phi_L
\end{align}
en donde $L=1.0$, $\rho=1.0$, $u=1.0$, $\Gamma=1/60$, $\phi_o=0.0$ , $\phi_L=1.0$. 

\subsection{Solución teórica adimensional}
\label{sec:solteo}
La ec.(\ref{eq:dif-conv}), tiene solución analítica y de forma adimensional es posible caracterizarla con un solo parámetro, llamado e el \emph{numero de Pecklet} (ver Figura.\ref{fig:solteo}). Definido de forma global $\mathbf{Pe}\,=\,\rho uL/\Gamma$ y de forma local mediante $\mathbf{Pe}_\Delta\,=\,\rho u\Delta x/\Gamma$. Utilizando el número $\mathbf{Pe}$, se obtiene
\begin{equation}
  \phi(x)\,=\,\phi_o+\frac{e^{x\mathbf{Pe}/L}-1}{e^{\mathbf{Pe}}-1}(\phi_L-\phi_0)
  \label{eq:solteo}
\end{equation}

\begin{figure}[H]
  \centering
  \includegraphics[width=9.0cm]{images/008SolucionTeorica.pdf}\\
  \label{fig:solteo}
  \caption{Solución analítica según el único parámetro adimensional $\mathbf{Pe}$}
\end{figure}

\subsection{Esquemas de prueba: ventajas y desventajas}
\label{sec:esquemas}
Los esquemas de diferencias finitas analizados en este apartado del documento son tres: \emph{esquemas de diferencias centrados} (CDS), \emph{esquemas up-wind} (UDS), CDS no-uniformes. En donde uniforme hace referencia al la uniformidad del tamaño de la separación de los nodos de la malla.

\subsubsection{Centrados Uniformes}
\label{sec:esq_centr}
Para solucionar la ec.(\ref{eq:dif-conv}), utilizando CDS, el estencil o kernel de convolución de la malla es:
\begin{equation}
  K_{cds}\,=\,[\,-\frac{\Gamma}{\Delta^2}-\frac{\rho u}{2\Delta},\quad 2\frac{\Gamma}{\Delta^2},\quad -\frac{\Gamma}{\Delta^2}+\frac{\rho u}{2\Delta}\,]
  \label{eq:Kcds}
\end{equation}
este tipo de esquem presenta oscilaciones alrededor de la solución analítica, que se reducen con la reducción tamaño de la malla. Este fenómeno de osciones se conose como \emph{oscilaciones artificiales}

\subsubsection{\emph{Up-wind} Uniformes}
\label{sec:esq_upwind}
Los esquemas up-wind, dependen de la dirección $u$ de la ecuación ec.(\ref{eq:dif-conv}), el estencil o kernel de convolución de la malla es:
\begin{align}
  K_{uds}\,=\,[&\quad -\frac{\Gamma}{\Delta^2}-\frac{\max(\rho u,0)}{\Delta},\notag\\
  &\quad +2\frac{\Gamma}{\Delta^2}-\frac{(\min(\rho u,0)-\max(\rho u,0))}{\Delta},\label{eq:Kuds}\\
  &\quad -\frac{\Gamma}{\Delta^2}+\frac{\min(\rho u,0)}{\Delta}\,]\notag
\end{align}
a diferenicia del CDS, este esquema up-wind tiene un poco más de complejidad para su implementación. Los resultados de este esquema generalmente producen unos coeficientes de difusión ficticios, este fenómeno númerico se conoce como \emph{difusividad artificial}, como se verá más adelante. Además como ventaja del esquema Up-wind, no existe oscilaciones artificiales.

\subsubsection{CDS no-uniformes}
\label{sec:cds-non-u}
Este esquema permite colocar más nodos, en lugares donde los cambios de los gradientes de la variable a resolver son elevados y reducirlos donde los gradientes son pequeños. La malla se puede generar de forma manual o utilzando una progresión geométrica.
\begin{align}
  K_{cds_{nu}}\,=\,2\Gamma[&\quad -\frac{1.0}{((x_{i+1}-x_{i-1})(x_i-x_{i-1}))}, \notag\\
         &\quad +\frac{1.0}{((x_{i+1}-x_{i-1})(x_i-x_{i-1}))} \notag\\
         &\qquad+ \frac{1.0}{((x_{i+1}-x_{i-1})(x_{i+1}-x_i))}, \label{eq:Knon-cds}\\
         &\quad -\frac{1.0}{((x_{i+1}-x_{i-1})(x_{i+1}-x_i))}\,] \notag\\
  +\,[&-\frac{\rho u}{(x_{i+1}-x_{i-1})}, 0.0, +\frac{\rho u}{(x_{i+1}-x_{i-1})}\,]\notag
  %K_stencil_CDS_fun = K_stencil_CDS_fun + (/ -rho*u/(x(i+1)-x(i-1)), 0.0_Long, +rho*u/(x(i+1)-x(i-1)) /)
\end{align}
es posible un esquema UDS no-uniforme, cambiando el vector añadido de la ec.(\ref{eq:Knon-cds}), este experimento se hizo en el código de este trabajo.

\subsection{Implementación}
\label{sec:imp_sol_dif_conv}

En el Listing.\ref{lst:impl_dif_conv}, se presenta una implementación modular que soluciona la ec.(\ref{eq:dif-conv}). Se dice modular por que este código me permite variar el tipo de kernel y a su vez el tipo de malla, generando la solución de los esquemas: CDS, UDS, non-CDS, non-UDS. La modularidad se logra al pasar como argumento una funcion de kernel \verb.K_fun(i,x,Gamma,rho,u)., que se declara mediante la interfaz de el código Listing.\ref{lst:impl_dif_conv}, líneas (15-25)

La solución consiste en obtener un sistema de la forma $\mathbf{D}\cdot\mathbf{\phi}\,=\,\mathbf{b}$, en donde la matriz $\mathbf{D}$ presenta la ventaja de ser tridiagonal. En el detalle la matriz se define como tres vectores (ver Listing.\ref{lst:impl_dif_conv}, líneas (9-10)), que conforman las tridiagonales. Para la inversion del sistema se utiliza el algoritmo de las librerias de LAPACK, DGTSV para precision \verb.double. (ver Listing.\ref{lst:impl_dif_conv}, líneas (52-53)). 

El manejo de las condiciones de Dirichlet, presentes en el problema se trabajan según Listing.\ref{lst:impl_dif_conv} en las líneas (27-41). Las del dominio interno Listing.\ref{lst:impl_dif_conv} en las líneas (43-49). El manejo de condiciones de frontera y de dominio interno ensamblan la matrix $\mathbf{D}$.

\begin{lstlisting}[caption={Algoritmo solución Difusión-Convección estacionario},label={lst:impl_dif_conv}]
  subroutine conv_difu_1D_non_uniform(x,phi,dim,&
                     K_fun,phi0,phiL,rho,u,Gamma)
    use parametros, only: Long        
    implicit none
    integer, intent(in) :: dim
    real(Long), dimension(dim), intent(in) :: x
    real(Long), dimension(dim), intent(out) :: phi
    real(Long), dimension(3):: K_stencil
    real(Long), dimension(dim-2) :: D
    real(Long), dimension(dim-3) :: DL, DU
    real(Long), intent(in) :: rho, u, Gamma
    real(Long) :: phi0, phiL
    integer :: i, info

    ! Declaracion funcion externa
    interface K_fcns
       function K_fun(i,x,Gamma,rho,u)
         use parametros
         implicit none
         integer :: i
         real(Long), dimension(:) :: x
         real(Long) :: Gamma, rho, u
         real(Long), dimension(3) :: K_fun
       end function K_fun
    end interface K_fcns

    ! boundaries
    ! x=0
    i = 2;
    K_stencil = K_fun(i,x,Gamma,rho,u)
    phi(1) = phi0
    phi(2) = -K_stencil(1)*phi0
    D(i-1) = K_stencil(2)
    DU(i-1) = K_stencil(3)
    ! x=L
    i = dim-1
    K_stencil = K_fun(i,x,Gamma,rho,u)
    DL(i-2) = K_stencil(1)
    D(i-1) = K_stencil(2)
    phi(dim-1) = -K_stencil(3)*phiL
    phi(dim) = phiL
    
    ! domain
    do i=3,dim-2
       K_stencil = K_fun(i,x,Gamma,rho,u)
       DL(i-2) = K_stencil(1)
       D(i-1) = K_stencil(2)
       DU(i-1) = K_stencil(3)        
    end do
    
    !    dgtsv(N,NRHS,DL,D,DU,b,LDB,info)
    call dgtsv(dim-2, 1, DL, D, DU, &
               phi(2:dim-1), dim-2, info)

  end subroutine conv_difu_1D_non_uniform
\end{lstlisting}

\subsection{Resultados}
\label{sec:res_sol_con_dif}

Efectos en malla uniforme con CDS y UDS, se pueden ver en la Figura.\ref{fig:sch_grid}, en donde se puede ver el efecto de oscilación artificial con los CDS y difusividad artifical en el caso de los UDS. En ambos casos a medida que se reduce el tamaño de la malla se mejora la aproximación de la solución.
\begin{figure}[H]
  \centering
  \includegraphics[width=9cm]{images/009Solu_Sch_Grid.pdf}\\
  \label{fig:sch_grid}
  \caption{Efectos en malla uniforme: con CDS, UDS y tamaño de malla}
\end{figure}

Efectos en malla no uniforme con CDS y UDS, dependen de la discretización de la malla. Nuevamente se puede observar en la Figura.\ref{fig:sch_n-ugrid}, que existe los fenomenos de oscilación y difusividad artificial asociados nuevamente a CDS y UDS, respectivamente. Sin embargo la precision de la malla no uniforme, no depende únicamente del número de puntos de la malla sino que depende de la ubicación de los puntos.
\begin{figure}[H]
  \centering
  \includegraphics[width=9cm]{images/010Solu_Sch_non-uGrid.pdf}\\
  \label{fig:sch_n-ugrid}
  \caption{Efectos en malla no uniforme: con CDS, UDS y tamaño de malla}
\end{figure}

En el siguiente ejemplo (Figura.\ref{fig:sch_n-ugrid-good}), se plantea una malla de solo 11 nodos que mejora, las anteiores mallas de 11 nodos, solo con el hecho de colocar los puntos de una mejor distribución. Aún así se puede ver los mismos efectos artificiales propios de cada esquema.
\begin{figure}[H]
  \centering
  \includegraphics[width=9.0cm]{images/011Solu_Sch_non-uGrid-Good.pdf}\\
  \label{fig:sch_n-ugrid-good}
  \caption{Buena malla no-uniforme}
\end{figure}

\section{Ecuación de difusión-convección transitoria 1D}
\label{sec:dif-conv-trans}
A continuación se analiza la ecuación de difusión-convección en estado estable para el caso unidimensional:
\begin{equation}
  \frac{\partial\,\rho\phi}{\partial t}\,+\,\frac{\partial\,\rho u \phi}{\partial x}\,=\,\frac{\partial}{\partial x}\,\left(\Gamma\frac{\partial\,\phi}{\partial x}\right)
  \label{eq:dif-conv-trans}
\end{equation}
sujeto a las condiciones de frontera,
\begin{align}
  \phi(x=0)\,&=\,\phi_o\\
  \phi(x=L)\,&=\,\phi_L
\end{align}
en donde $L=1.0$, $\rho=1.0$, $u=1.0$, $\Gamma=1/60$, $\phi_o=0.0$ , $\phi_L=1.0$. La condición incial para el caso transitorio satisface:
\begin{equation}
  \phi(x,0)\,=\,\phi_o\cos(\frac{\pi}{2}x)+\phi_L\sin(\frac{\pi}{2}x)
  \label{eq:dif-conv-trans-ci}
\end{equation}

\subsection{Solución Euler hacia adelante con CDS y non-CDS}
\label{sec:sol_ucds_noncds}

A continuación se muestra la solución espacio-temporal para dos mallas de 11 nodos, una uniforme y la otra no uniforme. En este caso se observa que las diferencias no uniformes producen difusividad ficticia.
\begin{figure}[H]
  \centering
  \includegraphics[width=9.0cm]{images/012Solu_Trans_FE_Grid11pts_11_1201.pdf}
  \caption{Solución temporal usando CDS $x=0.1$ y Forward Euler}
  \label{fig:trans_sol_cds_dx01}
\end{figure}

\begin{figure}[H]
  \centering
  \includegraphics[width=9.0cm]{images/013Solu_Trans_FE_nonUGrid11pts_11_1201.pdf}
  \caption{Solución temporal usando CDS $re=0.7$, $x=0.31$ y Forward Euler}
  \label{fig:trans_sol_non-cds_dx01}
\end{figure}

\subsection{Solución Runge-Kutta 4 con CDS}
\label{sec:sol_ucds_rk4_noncds}

En este caso se tomó una solución con una malla de 51 elementos durante los primeros 1.2 segundos de simulación
\begin{figure}[H]
  \centering
  \includegraphics[width=9.0cm]{images/014Solu_Trans_RK4_Grid_51_1201.pdf}
  \caption{Solución RK4, temporal usando CDS $dx=0.01$ y Forward Euler}
  \label{fig:trans_sol_rk4-cds_dx01}
\end{figure}

Finalmente la evolución temporal de tres puntos en el dominio: $x=0.5$, $x=0.7$ y $x=0.9$. Para cada uno de los metodos de Euler.
\begin{figure}[H]
  \centering
  \includegraphics[width=9.0cm]{images/015Solu_Trans_pts_t.pdf}
  \caption{Solución temporal usando CDS $dx=0.1$ y Forward Euler}
  \label{fig:trans_sol_pts_time_evol}
\end{figure}

\begin{figure*}
  \centering
  \includegraphics[width=\textwidth]{images/Figura_EFvsRK4dt0_05.pdf}
  \caption{Comentar}
  \label{fig:dtvsdxuG}
\end{figure*}

\begin{figure*}
  \centering
  \includegraphics[width=\textwidth]{images/Figura_nuEFvsnuRK4.pdf}
  \caption{Comentar}
  \label{fig:dtvsdxnuG}
\end{figure*}

\begin{figure*}
  \centering
  \includegraphics[width=\textwidth]{images/Figura_nuGrid.pdf}
  \caption{Comentar}
  \label{fig:dtvsdxnuGs}
\end{figure*}

\begin{figure*}
  \centering
  \includegraphics[width=\textwidth]{images/Figura_nuEF_dt_re_x0.pdf}
  \caption{Comentar}
  \label{fig:dtvsdxEFnuG}
\end{figure*}

%\section{Conclusiones}
%\label{sec:conclusiones}

%\section{Referencias}
%\label{sec:refs}

\end{multicols}

\newpage
\section{Anexos}
\label{sec:anexos}

\subsection{Calculos para la deducción de los esquemas de diferncias finitas}
\begin{multicols}{2}
Comenzando con las diferencias finitas de primer orden y la serie de Taylor que aproxima la función $f(x)$ al rededor del punto $x_i$, se tiene:

\begin{equation}
  f(x)\,=\,\sum^\infty_{k=0}\frac{(x-x_i)^k}{k!}\left.\frac{d^kf}{dx^k}\right|_{x_i}
  \label{eq:taylor}
\end{equation}
expandiendo unos términos
\begin{equation}
  \begin{split}
    f(x)&\,=\,f(x_i)\,+\,(x-x_i)\left.\frac{df}{dx}\right|_{x_i}\\
    &\quad+\,\frac{(x-x_i)^2}{2}\left.\frac{d^2f}{dx^2}\right|_{x_i}\\
    &\qquad+\,\frac{(x-x_i)^3}{6}\left.\frac{d^3f}{dx^3}\right|_{x_i}\,+\,\ldots
  \end{split}
  \label{eq:taylorexp}
\end{equation}
truncando la serie y aproximando un punto $x_{i+1}$ tal que $\Delta x_{i+1}:=x_{i+1}-x_i$, que además $\Delta x_{i+1}>0$, se tiene la siguiente approximación a la función,
\begin{equation}
  \begin{split}
    f(x_{i+1})&\,=\,f(x_i)\,+\,\Delta x_{i+1}\left.\frac{df}{dx}\right|_{x_i}\\
    &\quad+\,\frac{\Delta^2 x_{i+1}}{2}\left.\frac{d^2f}{dx^2}\right|_{x_i}\\
    &\qquad+\,\frac{\Delta^3 x_{i+1}}{6}\left.\frac{d^3f}{dx^3}\right|_{x_i}\,+\,\mathcal{O}(\Delta^4 x_{i+1})
  \end{split}
  \label{eq:taylorapprx}
\end{equation}
en donde se define $\Delta^n x_{i+1}\,:=\,(x_{i+1}-x_i)^n$ y la función del \emph{error de truncamiento de orden n} se define como $\mathcal{O}(\Delta^n x_{i+1})$ para todo $n\in[0,\infty)$.

\subsection{Serie de Taylor hacia adelante y hacia atrás}
\label{sec:backford}
 De forma similar la ec.\ref{eq:taylorapprx} se puede utilizar para calcular una \emph{aproximación hacia atrás}, en donde ahora $\,x_{i-1}-x_i\,<\,0$, cambia los signos de la ecuación de Taylor ec.\ref{eq:taylorapprx} de forma alterna, acontinuación
\begin{align}
  f_{i+1}&\,=\,f_i\,+\,\Delta_{i+1}f'_i\,+\,\frac{\Delta^2_{i+1}}{2}f''_i\,+\,\frac{\Delta^3_{i+1}}{6}f^3_i\,+\,\mathcal{O}^+(\Delta^4_{i+1})  \label{eq:tayloradelantesim}
  \\
 f_{i-1}&\,=\,f_i\,-\,\Delta_{i}f'_i\,+\,\frac{\Delta^2_{i}}{2}f''_i\,-\,\frac{\Delta^3_{i}}{6}f^3_i\,+\,\mathcal{O}^-(\Delta^4_{i}) 
  \label{eq:tayloratrassim}
\end{align}
en donde Taylor hacia adelante es ec.\ref{eq:tayloradelantesim} y Taylor hacia atrás es ec.\ref{eq:tayloratrassim},  la notación se simplifica aún más al eliminar la $x$, y los truncamientos se pueden definir como:
\begin{align}
  \mathcal{O}^+(\Delta^n_{i+1})&\,=\,\sum^\infty_{k=n}\,\frac{\Delta^k_{i+1}}{k!}f^{(k)}_i\label{eq:Otayloradelante}\\
  \mathcal{O}^-(\Delta^n_{i})&\,=\,\sum^\infty_{k=n}\,(-1)^n\frac{\Delta^k_i}{k!}f^{(k)}_i\label{eq:Otayloratras}
\end{align}

\subsection{Aproximación de la primera derivada simple}
\label{sec:primder}

Tomando la ecuación ec.\ref{eq:taylorapprx}, se procede a encontrar varias aproximaciones de la primera derivada $f'(x)=\frac{df}{dx}$, que varian según su orden de exactitud.

\subsubsection{Diferencia de primer orden hacia adelante}
\label{sec:dif1Da}
Truncando apartir del segundo término la ec.\ref{eq:tayloradelantesim},
\begin{equation}
  f'_i\,=\,\frac{f_{i+1}\,-\,f_i}{\Delta_{i+1}}\,-\,\frac{1}{\Delta_{i+1}}\mathcal{O}^+(\Delta^2_{i+1})
  \label{eq:apprx1dO2f}
\end{equation}
en la ecuación anterior, $\mathcal{O}^+/\Delta_{i+1}$ es el \emph{error de truncamiento de la primera derivada}, además la ec.\ref{eq:apprx1dO2f} se conoce como \emph{aproximación hacia adelante}.

\subsubsection{Diferencia de primer orden hacia atrás}
\label{sec:dif1Db}
Al obtener la diferencia de primer orden hacia atrás a partir de Taylor hacia atrás, queda: 
\begin{equation}
  f'_i\,=\,\frac{f_i\,-\,f_{i-1}}{\Delta_i}\,+\,\frac{1}{\Delta_i}\mathcal{O}^-(\Delta^2_i)
  \label{eq:apprx1dO2b}
\end{equation}

\subsubsection{Diferencia de primer orden centrada}
\label{sec:dif1Dc}
 Para tener una diferencia centrada basta con restar ec.\ref{eq:tayloratrassim} de ec.\ref{eq:tayloradelantesim}
\begin{equation}
  f'_i\,=\,\frac{f_{i+1}\,-\,f_{i-1}}{\Delta_{i+1}+\Delta_i}\,-\,\frac{1}{\Delta_{i+1}}\mathcal{O}^+(\Delta^2_{i+1})\,+\,\frac{1}{\Delta_i}\mathcal{O}^-(\Delta^2_i)
  \label{eq:apprx1dO2c}
\end{equation}
en el caso especial de que $\Delta=\Delta_{i+1}=\Delta_i$,
\begin{align}
  f'_i&\,=\,\frac{f_{i+1}\,-\,f_{i-1}}{2\Delta}\,-\,\frac{1}{\Delta}(\mathcal{O}^+(\Delta^2)\,-\,\mathcal{O}^-(\Delta^2))\notag\\
  &\,=\,\frac{f_{i+1}\,-\,f_{i-1}}{2\Delta}\,-\,\frac{1}{\Delta}\mathcal{O}^*(\Delta^3)
  \label{eq:apprx1dO2cEq}
\end{align}
donde 
\[
\mathcal{O}^*(\Delta^n_{i})\,=\,\sum^\infty_{k=n}\,(1-(-1)^k)\frac{\Delta^k_i}{k!}f^{k}_i\,=\,\sum^\infty_{k=n}\,2\frac{\Delta^{2k+1}_i}{(2k+1)!}f^{(2k+1)}_i
\]

\subsection{Aproximación de la primera derivada compuesta}
\label{sec:primdercompu}
A continucación se formarán primeras derivadas de orden 2, utilizando una malla uniforme.

\subsubsection{Diferencia de primer orden hacia adelante}
\label{sec:dif1D2Oa}
\begin{align}
  f_{i+2}&\,=\,f_i\,+\,2\Delta f'_i\,+\,\frac{4\Delta^2}{2}f''_i\,+\,\mathcal{O}^+((2\Delta)^3)  
  \label{eq:2TA}\\
  f_{i+1}&\,=\,f_i\,+\,\Delta f'_i\,+\,\frac{\Delta^2}{2}f''_i\,+\,\mathcal{O}^+(\Delta^3)  
  \label{eq:1TA}
\end{align}
sumando las ecuaciones ec.\ref{eq:2TA} y ec.\ref{eq:1TA}, y luego despejando $f'_i$,
\begin{align}
  f_{i+2}\,+\,f_{i+1}&\,=\,2f_i\,+\,3\Delta f'_i\,+\,\frac{5\Delta^2}{2}f''_i\,+\,\mathcal{O}^+((2\Delta)^3\,+\,\Delta^3)\notag\\
  f'_i&\,=\,\frac{f_{i+2}\,+\,f_{i+1}\,-\,2f_i}{3\Delta}\,-\,\frac{\mathcal{O}^+((2\Delta)^2\,+\,\Delta^2)}{3\Delta}
    \label{eq:2ordA}
\end{align}
ahora restando las ecuaciones cuatro veces la ec.\ref{eq:2TA} de ec.\ref{eq:1TA}, y luego despejando $f'_i$,
\begin{align}
  f_{i+2}\,-\,4f_{i+1}&\,=\,-3f_i\,-\,2\Delta f'_i\,+\,\mathcal{O}^+((2\Delta)^3\,-\,4\Delta^3)\notag\\
  f'_i&\,=\,\frac{-f_{i+2}\,+\,4f_{i+1}\,-\,3f_i}{2\Delta}\,+\,\frac{\mathcal{O}^+((2\Delta)^3\,-\,4\Delta^3)}{2\Delta}
    \label{eq:2ordAO3}
\end{align}

\subsubsection{Diferencia de primer orden hacia atrás}
\label{sec:dif1D2Ob}
\begin{align}
  f_{i-2}&\,=\,f_i\,-\,2\Delta f'_i\,+\,\frac{4\Delta^2}{2}f''_i\,+\,\mathcal{O}^-((2\Delta)^3)  
  \label{eq:2TB}\\
  f_{i-1}&\,=\,f_i\,-\,\Delta f'_i\,+\,\frac{\Delta^2}{2}f''_i\,+\,\mathcal{O}^-(\Delta^3)  
  \label{eq:1TB}
\end{align}
combinando las ecuaciones ec.\ref{eq:2TB} y ec.\ref{eq:1TB} para obtener un término de truncamiento del orden $\mathcal{O}(\Delta^3)$,
\begin{align}
  f_{i-2}\,-\,4f_{i-1}&\,=\,-3f_i\,+\,2\Delta f'_i\,+\,\mathcal{O}^-((2\Delta)^3\,-\,4\Delta^3)\notag\\
  f'_i&\,=\,\frac{3f_{i}\,-\,4f_{i-1}\,+\,f_{i-2}}{2\Delta}\,-\,\frac{\mathcal{O}^-((2\Delta)^3\,-\,4\Delta^3)}{2\Delta}
    \label{eq:2ordB}
\end{align}

\subsubsection{Diferencia de primer orden centradas}
\label{sec:dif1D2Oc}
\begin{align}
  f_{i+2}\,-\,4f_{i+1}-f_{i-2}\,+\,4f_{i-1}&\,=\,-\,4\Delta f'_i\notag\\
  &\quad+\,\mathcal{O}^+((2\Delta)^3\,-\,4\Delta^3)\notag\\
  &\qquad-\,\mathcal{O}^-((2\Delta)^3\,-\,4\Delta^3)\notag
\end{align}

\begin{equation}
  f'_i\,=\,\frac{-f_{i+2}\,+\,4f_{i+1}\,-\,4f_{i-1}\,+\,f_{i-2}}{4\Delta}\,-\,\frac{\mathcal{O}^*((2\Delta)^3\,-\,4\Delta^3)}{4\Delta}
  \label{eq:2ordC}
\end{equation}

\emph{prueba:}
{\tiny
  \begin{align}
    -f_{i+2}&\,=\,-f_i
              \,-\,2\Delta f'_i
              \,-\,\frac{4\Delta^2}{2}f''_i
              \,-\,\frac{8\Delta^3}{6}f^3_i
              \,-\,\frac{16\Delta^4}{24}f^4_i
              \,-\,\frac{32\Delta^5}{125}f^5_i  
              \,-\,\ldots  
              \notag\\
    4f_{i+1}&\,=\,4f_i
              \,+\,4\Delta f'_i
              \,+\,4\frac{\Delta^2}{2}f''_i
              \,+\,4\frac{\Delta^3}{6}f^3_i
              \,+\,4\frac{\Delta^4}{24}f^4_i
              \,+\,4\frac{\Delta^5}{125}f^5_i  
              \,+\,\ldots  
              \notag\\
    -4f_{i-1}&\,=\,-4f_i
               \,+\,4\Delta f'_i
               \,-\,4\frac{\Delta^2}{2}f''_i
               \,+\,4\frac{\Delta^3}{6}f^3_i
               \,-\,4\frac{\Delta^4}{24}f^4_i
               \,+\,4\frac{\Delta^5}{125}f^5_i  
               \,-\,\ldots  
               \notag\\
    f_{i-2}&\,=\,f_i
             \,-\,2\Delta f'_i
             \,+\,\frac{4\Delta^2}{2}f''_i
             \,-\,\frac{8\Delta^3}{6}f^3_i
             \,+\,\frac{16\Delta^4}{24}f^4_i
             \,-\,\frac{32\Delta^5}{125}f^5_i  
             \,+\,\ldots
             \notag
  \end{align}
}
sumando y simplificando 
{\tiny
  % 1,3,5,...= 2n+1, for all n in N
  % 3,5,7,...= 2n+3, for all n in N {0,1,2}
  \begin{align}
    -f_{i+2}+4f_{i+1}-4f_{i-1}+f_{i-2}\,&=\,0f_i
                                          +4\Delta_if'_i
                                          +0f''
                                          -\frac{8\Delta^3}{6}f^3_i
                                          +0f^4
                                          -\frac{56\Delta^5}{125}f^5_i\notag\\
    -f_{i+2}+4f_{i+1}-4f_{i-1}+f_{i-2}\,&=\,+4\Delta_if'_i
                                          -\frac{8\Delta^3}{6}f^3_i
                                          -\frac{56\Delta^5}{125}f^5_
                                          -\ldots\notag\\
    -f_{i+2}+4f_{i+1}-4f_{i-1}+f_{i-2}\,&=\,+4\Delta_if'_i
                                          -\sum^\infty_{k=0}\frac{(2^{2k+4}-8)\Delta^{2k+3}}{(2k+3)!}f^{2k+3}_i\notag\\
    -f_{i+2}+4f_{i+1}-4f_{i-1}+f_{i-2}\,&=\,+4\Delta_if'_i
                                          -\mathcal{O}(\Delta^3)
  \end{align}
}

\subsubsection{Diferencia de primer orden centradas $\mathcal{O}(\Delta^5)$}
\label{sec:dif1D2Od}
El problema general es sumar las siguientes aproximaciones,
{\tiny
  \begin{align}
    af_{i+2}&\,=\,af_i
              \,+\,a2\Delta f'_i
              \,+\,a\frac{4\Delta^2}{2}f''_i
              \,+\,a\frac{8\Delta^3}{6}f^3_i
              \,+\,a\frac{16\Delta^4}{24}f^4_i
              \,+\,a\frac{32\Delta^5}{125}f^5_i  
              \,+\,\ldots  
              \notag\\
    bf_{i+1}&\,=\,bf_i
              \,+\,b\Delta f'_i
              \,+\,b\frac{\Delta^2}{2}f''_i
              \,+\,b\frac{\Delta^3}{6}f^3_i
              \,+\,b\frac{\Delta^4}{24}f^4_i
              \,+\,b\frac{\Delta^5}{125}f^5_i  
              \,+\,\ldots  
              \notag\\
    cf_{i-1}&\,=\,cf_i
               \,-\,c\Delta f'_i
               \,+\,c\frac{\Delta^2}{2}f''_i
               \,-\,c\frac{\Delta^3}{6}f^3_i
               \,+\,c\frac{\Delta^4}{24}f^4_i
               \,-\,c\frac{\Delta^5}{125}f^5_i  
               \,+\,\ldots  
               \notag\\
    df_{i-2}&\,=\,df_i
             \,-\,d2\Delta f'_i
             \,+\,d\frac{4\Delta^2}{2}f''_i
             \,-\,d\frac{8\Delta^3}{6}f^3_i
             \,+\,d\frac{16\Delta^4}{24}f^4_i
             \,-\,d\frac{32\Delta^5}{125}f^5_i  
             \,+\,\ldots
             \notag
  \end{align}
}
tales que
{%\small
  \begin{align}
    a
    \,+\,b
    \,+\,c
    \,+\,d
    &\,=\,\alpha \notag\\
    2a
    \,+\, b
    \,-\, c
    \,-\,2d
    &\,=\,\beta \notag\\
    a\frac{4}{2}
    \,+\,b\frac{1}{2}
    \,+\,c\frac{1}{2}
    \,+\,d\frac{4}{2}
                         &\,=\,0 \notag\\
    a\frac{8}{6}
    \,+\,b\frac{1}{6}
    \,-\,c\frac{1}{6}
    \,-\,d\frac{8}{6}
                         &\,=\,0 \notag\\  
    a\frac{16}{24}
    \,+\,b\frac{1}{24}
    \,+\,c\frac{1}{24}
    \,+\,d\frac{16}{24}
                         &\,=\,0 \notag
%    a\frac{32\Delta^5}{125}f^5_i  
%    \,+\,b\frac{\Delta^5}{125}f^5_i  
%    \,-\,c\frac{\Delta^5}{125}f^5_i  
%    \,-\,d\frac{32\Delta^5}{125}f^5_i  
%                         &\,=\,0 \notag 
  \end{align}
}
\end{multicols}

\subsubsection{Algoritmo de cálculo de error optimizado en espacio}
\label{sec:opt_alg}

El algoritmo de Listing.\ref{lst:opt_alg}, detalla la implementación utilizada el documento para la investigación del error de redondeo y truncamiento encontrado en los esquemas de direncias finitas.

\begin{multicols}{2}
\begin{lstlisting}[caption={Algoritmo del cálculo de las derivadas},label={lst:opt_alg}]
subroutine errors_kernel(dim,K_domain,Kd_cols,&
       K_boundary,Kb_cols,Kb_rows,&
       err_max,err_b_max,errd2_max,errd2_b_max)
  use parametros, only: Long, pi, k1, k2
  implicit none
  real(Long), intent(in), dimension(:) :: K_domain
  real(Long), intent(in), dimension(:,:) :: K_boundary
  integer, intent(in) :: dim, Kd_cols, Kb_cols, Kb_rows
  real(Long), intent(out) :: err_max, err_b_max
  real(Long), intent(out) :: errd2_max, errd2_b_max
  real(Long), dimension(Kd_cols) :: x_tmp, f_tmp, df_tmp
  real(Long) :: err_i, errd2_i 
  real(Long) :: fun_i, dfun_i, d2fun_i
  integer :: i, j, k

  err_max = -1; err_b_max = -1
  errd2_max = -1; errd2_b_max = -1

  x_tmp(1:Kb_cols) = (/ (2*pi*(i-1)/(dim-1)-pi, ! 
                          i=1,Kb_cols) /)
  f_tmp = exp(sin(x_tmp)) + 0.5*cos(k1*x_tmp) &
          - 0.8*sin(k2*x_tmp)

  ! left boundary
  do i=1,Kb_rows
     ! first derivative
     dfun_i = dot_product(K_boundary(i,:), &
                          f_tmp(1:Kb_cols))
     df_tmp(i) = dfun_i;
     !postprocess
     err_i = abs(dfun(1.0_Long*i)-dfun_i)
     if ( err_b_max < err_i ) err_b_max=err_i
  end do

  ! domain
  do i=Kb_rows+1,dim-Kb_rows
     ! first derivative
     if (i>Kb_rows+1) &
        call push_back(f_tmp, &
                       fun(1.0_Long*(i+Kb_rows)), &
                       Kd_cols)
     dfun_i = dot_product(K_domain,f_tmp)
     ! second derivative
     if (i<Kd_cols) then
        df_tmp(i) = dfun_i;
     elseif (i==Kd_cols) then
        df_tmp(i) = dfun_i;
        do j=1,Kb_rows ! boundary
           d2fun_i = dot_product(K_boundary(j,:),df_tmp)
           errd2_i = abs(d2fun(1.0_Long*j)-d2fun_i)
           if ( errd2_b_max < errd2_i ) errd2_b_max=errd2_i
        end do
        d2fun_i = dot_product(K_domain,df_tmp)
        errd2_i = abs(d2fun(1.0_Long*(i-Kb_rows))-d2fun_i)
        if ( errd2_b_max < errd2_i ) errd2_b_max=errd2_i
     else ! domain
        call push_back(df_tmp,dfun_i,Kb_cols)
        d2fun_i = dot_product(K_domain,df_tmp)
        errd2_i = abs(d2fun(1.0_Long*(i-Kb_rows))-d2fun_i)
        if ( errd2_max < errd2_i ) errd2_max=errd2_i
     end if
     ! postprocess
     err_i = abs(dfun(1.0_Long*i)-dfun_i)
     if ( err_max < err_i ) err_max=err_i
  end do

  ! right boundary
  do i=dim-Kb_rows+1,dim
     ! first derivative
     dfun_i = dot_product(-K_boundary(dim-i+1,:), &
                           f_tmp(Kb_cols:1:-1))
     ! second derivative
     call push_back(df_tmp,dfun_i,Kb_cols)
     d2fun_i = dot_product(K_domain,df_tmp)
     errd2_i = abs(d2fun(1.0_Long*(i-Kb_rows))-d2fun_i)
     if ( errd2_max < errd2_i ) errd2_max=errd2_i
     ! postprocess
     err_i = abs(dfun(1.0_Long*i)-dfun_i)
     if ( err_b_max < err_i ) err_b_max=err_i
  end do
  ! second derivative
  do j=1,Kb_rows
     d2fun_i = dot_product( & 
                    -K_boundary(Kb_rows-j+1,Kb_cols:1:-1), &
                    df_tmp)
     errd2_i = abs(d2fun(1.0_Long*(dim-Kb_rows+j))-d2fun_i)
     if ( errd2_b_max < errd2_i ) errd2_b_max=errd2_i
  end do
    
end subroutine errors_kernel
\end{lstlisting}
\end{multicols}

\subsubsection{Algoritmo esquemas temporales}
\label{sec:dif_conv_trans_alg}

Euler hacia adelante, Runge-Kutta con CDS y non-uniform CDS:
\begin{multicols}{2}
\begin{lstlisting}[caption={Algoritmo Forward-Euler, Runge-Kutta},label={lst:trans_alg}]
program difusion_conveccion_trans
  use diferencias_finitas
  implicit none
  integer, parameter :: dim = 51, num_plts = 11
  real*8, parameter :: pi = acos(-1.0D0)
  real*8, dimension(dim) :: x
  real*8, dimension(dim) :: phi_exact, phi_tmp, &
                    phi_CDS, phi_non_CDS
  real*8, dimension(dim) :: K1, K2, K3, K4
  real*8, dimension(dim,num_plts) :: data
  real*8, dimension(dim-2,3) :: D
  real*8 :: Gamma = 1.0D0/40.0D0, L = 1.0D0, &
                    rho = 1.0D0, u = 1.0D0
  real*8 :: phi0 = 0.0D0, phiL = 1.0D0
  real*8 :: dx, dt = 0.001, t = 0.0D0, tmax = 1.2D0
  real*8 :: re = 0.7, dx0 = 0.31
  real*8 :: Pe
  real*8 :: alpha, beta
  integer :: i, j, k, dt_plot = tmax/dt/(num_plt-1)
  integer :: info

  Pe = rho*u*L/Gamma
  dx = L/(dim-1)

  ! mesh and exact phi info
  x = (/ (real(i)/(dim-1), i=0,dim-1) /)

  phi_exact = phi0 &
          + (exp(Pe*x/L)-1)/(exp(Pe)-1)*(phiL-phi0)

  phi_CDS = phi0*cos(pi/2*x)+phiL*sin(pi/2*x)

  ! 1. CDS - Forward Euler
  alpha = Gamma/dx**2; beta = rho*u/2/dx;
  j = dt_plot; k = 1;
  do
     phi_tmp = phi_CDS
     if (j == dt_plot) then
        j = 0
        data(:,k) = phi_CDS
        k = k+1
     end if
     j = j+1
     phi_CDS(2:dim-1) = phi_tmp(2:dim-1) &
                  +dt*((alpha+beta)*phi_tmp(1:dim-2) &
                  -2*alpha*phi_tmp(2:dim-1) &
                  +(alpha-beta)*phi_tmp(3:dim))
     if (t>=tmax) exit
     t = t+dt
  end do

  ! 2. CDS - Runge-Kutta
  alpha = Gamma/dx**2; beta = rho*u/2/dx;
  j = dt_plot; k = 1;
  do
     phi_tmp = phi_CDS ! *** optimizar
     if (j == dt_plot) then
        j = 0
        data(:,k) = phi_CDS
        k = k+1
     end if
     j = j+1
     ! phi_tmp(i)
     do i = 2,dim-1
        K1(i) = dt*( (alpha+beta)*phi_tmp(i-1) &
             - 2*alpha*phi_tmp(i) &
             + (alpha-beta)*phi_tmp(i+1) ) 
     end do
     do i = 2,dim-1
        K2(i) = dt*( (alpha+beta)*( phi_tmp(i-1) &
                      +0.5*K1(i-1) ) &
             - 2*alpha*( phi_tmp(i)+0.5*K1(i) ) &
             + (alpha-beta)*( phi_tmp(i+1) &
                              +0.5*K1(i+1) ) ) 
     end do
     do i = 2,dim-1
        K3(i) = dt*( (alpha+beta)*( phi_tmp(i-1) &
                                    +0.5*K2(i-1) ) &
             - 2*alpha*( phi_tmp(i)+0.5*K2(i) ) &
             + (alpha-beta)*( phi_tmp(i+1)+0.5*K2(i+1) ) ) 
     end do
     do i = 2,dim-1
        K4(i) = dt*( (alpha+beta)*( phi_tmp(i-1) &
                                    +K3(i-1) ) &
             - 2*alpha*( phi_tmp(i)+K3(i) ) &
             + (alpha-beta)*( phi_tmp(i+1)+K3(i+1) ) ) 
     end do
     phi_CDS = phi_tmp + (1./6.)*(K1 + 2.*K2 + 2.*K3 + K4)
     if (t>=tmax) exit
     t = t+dt
  end do

  ! 3. non-CDS - Forward Euler
  ! mesh and exact phi info
  x(1) = 0.0; x(2) = dx0
  do i=3,dim
     x(i) = (x(i-1)-x(i-2))*re+x(i-1)
  end do
  x(dim) = 1.0D0
  !
  phi_exact = phi0 + (exp(Pe*x/L)-1)/(exp(Pe)-1)*(phiL-phi0)

  phi_non_CDS = phi0*cos(pi/2*x)+phiL*sin(pi/2*x)

  do i = 2,dim-1
     D(i-1,:) = 2*Gamma*(/ -1.0/((x(i+1)&
                             -x(i-1))*(x(i)&
                             -x(i-1))), &
          +1.0/((x(i+1)-x(i-1))*(x(i)-x(i-1))) &
            + 1.0/((x(i+1)-x(i-1))*(x(i+1)-x(i))), &
          -1.0/((x(i+1)-x(i-1))*(x(i+1)-x(i))) /)
     D(i-1,:) = D(i-1,:) &
               + (/ -rho*u/(x(i+1)-x(i-1)), &
                    0.0D0, &
                    +rho*u/(x(i+1)-x(i-1)) /)
  end do
  alpha = Gamma/dx**2; beta = rho*u/2/dx;
  j = dt_plot; k = 1;
  do
     phi_tmp = phi_non_CDS
     if (j == dt_plot) then
        j = 0
        data(:,k) = phi_CDS
        k = k+1
     end if
     j = j+1
     do i = 2,dim-1
        phi_non_CDS(i) = phi_tmp(i) &
             + dt*( D(i-1,1)*phi_tmp(i-1) &
             - D(i-1,2)*phi_tmp(i) &
             + D(i-1,3)*phi_tmp(i+1) ) 
     end do
     if (t>=tmax) exit
     t = t+dt
  end do

end program difusion_conveccion_trans
\end{lstlisting}
\end{multicols}

\end{document}
