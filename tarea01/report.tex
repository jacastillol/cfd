\documentclass[9pt,technote,twoside,letterpaper,twocolumn]{IEEEtran}

\usepackage{amsmath}
\usepackage[utf8]{inputenc}
\usepackage{graphicx}

\title{Introducción a las Diferencias Finitas}
\author{Jaime~Andrés~Castillo-León}

\begin{document}
\maketitle

\section{Exactitud de esquemas de diferencias finitas}
\label{sec:exac}
En el siguiente ejemplo se aproxima la primera y segunda derivada de una función $f(x)$ analitica conocida.
\begin{equation}
  f(x)\,=\,\exp(\sin(x))\,+\,0.5\cos(\kappa_1x)\,-\,0.8\sin(\kappa_2x)
  \label{eq:fun}
\end{equation}

\begin{align}
  \frac{df}{dx}\,&=\,\cos(x)\exp(\sin(x))\notag\\
&\quad-\,0.5\kappa_1\sin(\kappa_1x)\,-\,0.8\kappa_2\cos(\kappa_2x)
  \label{eq:dfun}
\end{align}

\begin{align}
  \frac{d^2f}{dx^2}\,&=\,(\cos^2(x)-\,\sin(x))\exp(\sin(x))\notag\\
&\quad-\,0.5\kappa^2_1\cos(\kappa_1x)\,+\,0.8\kappa^2_2\sin(\kappa_2x)
  \label{eq:d2fun}
\end{align}

Comenzando con las diferencias finitas de primer orden y la serie de Taylor que aproxima la función $f(x)$ al rededor del punto $x_i$, se tiene:

\begin{equation}
  f(x)\,=\,\sum^\infty_{k=0}\frac{(x-x_i)^k}{k!}\left.\frac{d^kf}{dx^k}\right|_{x_i}
  \label{eq:taylor}
\end{equation}
expandiendo unos términos
\begin{equation}
  \begin{split}
    f(x)&\,=\,f(x_i)\,+\,(x-x_i)\left.\frac{df}{dx}\right|_{x_i}\\
    &\quad+\,\frac{(x-x_i)^2}{2}\left.\frac{d^2f}{dx^2}\right|_{x_i}\\
    &\qquad+\,\frac{(x-x_i)^3}{6}\left.\frac{d^3f}{dx^3}\right|_{x_i}\,+\,\ldots
  \end{split}
  \label{eq:taylorexp}
\end{equation}
truncando la serie y aproximando un punto $x_{i+1}$ tal que $\Delta x_{i+1}:=x_{i+1}-x_i$, que además $\Delta x_{i+1}>0$, se tiene la siguiente approximación a la función,
\begin{equation}
  \begin{split}
    f(x_{i+1})&\,=\,f(x_i)\,+\,\Delta x_{i+1}\left.\frac{df}{dx}\right|_{x_i}\\
    &\quad+\,\frac{\Delta^2 x_{i+1}}{2}\left.\frac{d^2f}{dx^2}\right|_{x_i}\\
    &\qquad+\,\frac{\Delta^3 x_{i+1}}{6}\left.\frac{d^3f}{dx^3}\right|_{x_i}\,+\,\mathcal{O}(\Delta^4 x_{i+1})
  \end{split}
  \label{eq:taylorapprx}
\end{equation}
en donde se define $\Delta^n x_{i+1}\,:=\,(x_{i+1}-x_i)^n$ y la función del \emph{error de truncamiento de orden n} se define como $\mathcal{O}(\Delta^n x_{i+1})$ para todo $n\in[0,\infty)$.

\subsection{Serie de Taylor hacia adelante y hacia atrás}
\label{sec:backford}
 De forma similar la ec.\ref{eq:taylorapprx} se puede utilizar para calcular una \emph{aproximación hacia atrás}, en donde ahora $\,x_{i-1}-x_i\,<\,0$, cambia los signos de la ecuación de Taylor ec.\ref{eq:taylorapprx} de forma alterna, acontinuación
\begin{align}
  f_{i+1}&\,=\,f_i\,+\,\Delta_{i+1}f'_i\,+\,\frac{\Delta^2_{i+1}}{2}f''_i\,+\,\frac{\Delta^3_{i+1}}{6}f^3_i\,+\,\mathcal{O}^+(\Delta^4_{i+1})  \label{eq:tayloradelantesim}
  \\
 f_{i-1}&\,=\,f_i\,-\,\Delta_{i}f'_i\,+\,\frac{\Delta^2_{i}}{2}f''_i\,-\,\frac{\Delta^3_{i}}{6}f^3_i\,+\,\mathcal{O}^-(\Delta^4_{i}) 
  \label{eq:tayloratrassim}
\end{align}
en donde Taylor hacia adelante es ec.\ref{eq:tayloradelantesim} y Taylor hacia atrás es ec.\ref{eq:tayloratrassim},  la notación se simplifica aún más al eliminar la $x$, y los truncamientos se pueden definir como:
\begin{align}
  \mathcal{O}^+(\Delta^n_{i+1})&\,=\,\sum^\infty_{k=n}\,\frac{\Delta^k_{i+1}}{k!}f^{(k)}_i\label{eq:Otayloradelante}\\
  \mathcal{O}^-(\Delta^n_{i})&\,=\,\sum^\infty_{k=n}\,(-1)^n\frac{\Delta^k_i}{k!}f^{(k)}_i\label{eq:Otayloratras}
\end{align}

\subsection{Aproximación de la primera derivada simple}
\label{sec:primder}

Tomando la ecuación ec.\ref{eq:taylorapprx}, se procede a encontrar varias aproximaciones de la primera derivada $f'(x)=\frac{df}{dx}$, que varian según su orden de exactitud.

\subsubsection{Diferencia de primer orden hacia adelante}
\label{sec:dif1Da}
Truncando apartir del segundo término la ec.\ref{eq:tayloradelantesim},
\begin{equation}
  f'_i\,=\,\frac{f_{i+1}\,-\,f_i}{\Delta_{i+1}}\,-\,\frac{1}{\Delta_{i+1}}\mathcal{O}^+(\Delta^2_{i+1})
  \label{eq:apprx1dO2f}
\end{equation}
en la ecuación anterior, $\mathcal{O}^+/\Delta_{i+1}$ es el \emph{error de truncamiento de la primera derivada}, además la ec.\ref{eq:apprx1dO2f} se conoce como \emph{aproximación hacia adelante}.

\subsubsection{Diferencia de primer orden hacia atrás}
\label{sec:dif1Db}
Al obtener la diferencia de primer orden hacia atrás a partir de Taylor hacia atrás, queda: 
\begin{equation}
  f'_i\,=\,\frac{f_i\,-\,f_{i-1}}{\Delta_i}\,+\,\frac{1}{\Delta_i}\mathcal{O}^-(\Delta^2_i)
  \label{eq:apprx1dO2b}
\end{equation}

\subsubsection{Diferencia de primer orden centrada}
\label{sec:dif1Dc}
 Para tener una diferencia centrada basta con restar ec.\ref{eq:tayloratrassim} de ec.\ref{eq:tayloradelantesim}
\begin{equation}
  f'_i\,=\,\frac{f_{i+1}\,-\,f_{i-1}}{\Delta_{i+1}+\Delta_i}\,-\,\frac{1}{\Delta_{i+1}}\mathcal{O}^+(\Delta^2_{i+1})\,+\,\frac{1}{\Delta_i}\mathcal{O}^-(\Delta^2_i)
  \label{eq:apprx1dO2c}
\end{equation}
en el caso especial de que $\Delta=\Delta_{i+1}=\Delta_i$,
\begin{align}
  f'_i&\,=\,\frac{f_{i+1}\,-\,f_{i-1}}{2\Delta}\,-\,\frac{1}{\Delta}(\mathcal{O}^+(\Delta^2)\,-\,\mathcal{O}^-(\Delta^2))\notag\\
  &\,=\,\frac{f_{i+1}\,-\,f_{i-1}}{2\Delta}\,-\,\frac{1}{\Delta}\mathcal{O}^*(\Delta^3)
  \label{eq:apprx1dO2cEq}
\end{align}
donde 
\[
\mathcal{O}^*(\Delta^n_{i})\,=\,\sum^\infty_{k=n}\,(1-(-1)^k)\frac{\Delta^k_i}{k!}f^{k}_i\,=\,\sum^\infty_{k=n}\,2\frac{\Delta^{2k+1}_i}{(2k+1)!}f^{(2k+1)}_i
\]

\subsection{Aproximación de la primera derivada compuesta}
\label{sec:primdercompu}
A continucación se formarán primeras derivadas de orden 2, utilizando una malla uniforme.

\subsubsection{Diferencia de primer orden hacia adelante}
\label{sec:dif1D2Oa}
\begin{align}
  f_{i+2}&\,=\,f_i\,+\,2\Delta f'_i\,+\,\frac{4\Delta^2}{2}f''_i\,+\,\mathcal{O}^+((2\Delta)^3)  
  \label{eq:2TA}\\
  f_{i+1}&\,=\,f_i\,+\,\Delta f'_i\,+\,\frac{\Delta^2}{2}f''_i\,+\,\mathcal{O}^+(\Delta^3)  
  \label{eq:1TA}
\end{align}
sumando las ecuaciones ec.\ref{eq:2TA} y ec.\ref{eq:1TA}, y luego despejando $f'_i$,
\begin{align}
  f_{i+2}\,+\,f_{i+1}&\,=\,2f_i\,+\,3\Delta f'_i\,+\,\frac{5\Delta^2}{2}f''_i\,+\,\mathcal{O}^+((2\Delta)^3\,+\,\Delta^3)\notag\\
  f'_i&\,=\,\frac{f_{i+2}\,+\,f_{i+1}\,-\,2f_i}{3\Delta}\,-\,\frac{\mathcal{O}^+((2\Delta)^2\,+\,\Delta^2)}{3\Delta}
    \label{eq:2ordA}
\end{align}
ahora restando las ecuaciones cuatro veces la ec.\ref{eq:2TA} de ec.\ref{eq:1TA}, y luego despejando $f'_i$,
\begin{align}
  f_{i+2}\,-\,4f_{i+1}&\,=\,-3f_i\,-\,2\Delta f'_i\,+\,\mathcal{O}^+((2\Delta)^3\,-\,4\Delta^3)\notag\\
  f'_i&\,=\,\frac{-f_{i+2}\,+\,4f_{i+1}\,-\,3f_i}{2\Delta}\,+\,\frac{\mathcal{O}^+((2\Delta)^3\,-\,4\Delta^3)}{2\Delta}
    \label{eq:2ordAO3}
\end{align}

\subsubsection{Diferencia de primer orden hacia atrás}
\label{sec:dif1D2Ob}
\begin{align}
  f_{i-2}&\,=\,f_i\,-\,2\Delta f'_i\,+\,\frac{4\Delta^2}{2}f''_i\,+\,\mathcal{O}^-((2\Delta)^3)  
  \label{eq:2TB}\\
  f_{i-1}&\,=\,f_i\,-\,\Delta f'_i\,+\,\frac{\Delta^2}{2}f''_i\,+\,\mathcal{O}^-(\Delta^3)  
  \label{eq:1TB}
\end{align}
combinando las ecuaciones ec.\ref{eq:2TB} y ec.\ref{eq:1TB} para obtener un término de truncamiento del orden $\mathcal{O}(\Delta^3)$,
\begin{align}
  f_{i-2}\,-\,4f_{i-1}&\,=\,-3f_i\,+\,2\Delta f'_i\,+\,\mathcal{O}^-((2\Delta)^3\,-\,4\Delta^3)\notag\\
  f'_i&\,=\,\frac{3f_{i}\,-\,4f_{i-1}\,+\,f_{i-2}}{2\Delta}\,-\,\frac{\mathcal{O}^-((2\Delta)^3\,-\,4\Delta^3)}{2\Delta}
    \label{eq:2ordB}
\end{align}

\subsubsection{Diferencia de primer orden centradas}
\label{sec:dif1D2Oc}
\begin{align}
  f_{i+2}\,-\,4f_{i+1}-f_{i-2}\,+\,4f_{i-1}&\,=\,-\,4\Delta f'_i\notag\\
  &\quad+\,\mathcal{O}^+((2\Delta)^3\,-\,4\Delta^3)\notag\\
  &\qquad-\,\mathcal{O}^-((2\Delta)^3\,-\,4\Delta^3)\notag
\end{align}

\begin{equation}
  f'_i\,=\,\frac{-f_{i+2}\,+\,4f_{i+1}\,-\,4f_{i-1}\,+\,f_{i-2}}{4\Delta}\,-\,\frac{\mathcal{O}^*((2\Delta)^3\,-\,4\Delta^3)}{4\Delta}
  \label{eq:2ordC}
\end{equation}

\emph{prueba:}
{\tiny
  \begin{align}
    -f_{i+2}&\,=\,-f_i
              \,-\,2\Delta f'_i
              \,-\,\frac{4\Delta^2}{2}f''_i
              \,-\,\frac{8\Delta^3}{6}f^3_i
              \,-\,\frac{16\Delta^4}{24}f^4_i
              \,-\,\frac{32\Delta^5}{125}f^5_i  
              \,-\,\ldots  
              \notag\\
    4f_{i+1}&\,=\,4f_i
              \,+\,4\Delta f'_i
              \,+\,4\frac{\Delta^2}{2}f''_i
              \,+\,4\frac{\Delta^3}{6}f^3_i
              \,+\,4\frac{\Delta^4}{24}f^4_i
              \,+\,4\frac{\Delta^5}{125}f^5_i  
              \,+\,\ldots  
              \notag\\
    -4f_{i-1}&\,=\,-4f_i
               \,+\,4\Delta f'_i
               \,-\,4\frac{\Delta^2}{2}f''_i
               \,+\,4\frac{\Delta^3}{6}f^3_i
               \,-\,4\frac{\Delta^4}{24}f^4_i
               \,+\,4\frac{\Delta^5}{125}f^5_i  
               \,-\,\ldots  
               \notag\\
    f_{i-2}&\,=\,f_i
             \,-\,2\Delta f'_i
             \,+\,\frac{4\Delta^2}{2}f''_i
             \,-\,\frac{8\Delta^3}{6}f^3_i
             \,+\,\frac{16\Delta^4}{24}f^4_i
             \,-\,\frac{32\Delta^5}{125}f^5_i  
             \,+\,\ldots
             \notag
  \end{align}
}
sumando y simplificando 
{\tiny
  % 1,3,5,...= 2n+1, for all n in N
  % 3,5,7,...= 2n+3, for all n in N {0,1,2}
  \begin{align}
    -f_{i+2}+4f_{i+1}-4f_{i-1}+f_{i-2}\,&=\,0f_i
                                          +4\Delta_if'_i
                                          +0f''
                                          -\frac{8\Delta^3}{6}f^3_i
                                          +0f^4
                                          -\frac{56\Delta^5}{125}f^5_i\notag\\
    -f_{i+2}+4f_{i+1}-4f_{i-1}+f_{i-2}\,&=\,+4\Delta_if'_i
                                          -\frac{8\Delta^3}{6}f^3_i
                                          -\frac{56\Delta^5}{125}f^5_
                                          -\ldots\notag\\
    -f_{i+2}+4f_{i+1}-4f_{i-1}+f_{i-2}\,&=\,+4\Delta_if'_i
                                          -\sum^\infty_{k=0}\frac{(2^{2k+4}-8)\Delta^{2k+3}}{(2k+3)!}f^{2k+3}_i\notag\\
    -f_{i+2}+4f_{i+1}-4f_{i-1}+f_{i-2}\,&=\,+4\Delta_if'_i
                                          -\mathcal{O}(\Delta^3)
  \end{align}
}

\subsubsection{Diferencia de primer orden centradas $\mathcal{O}(\Delta^5)$}
\label{sec:dif1D2Od}
El problema general es sumar las siguientes aproximaciones,
{\tiny
  \begin{align}
    af_{i+2}&\,=\,af_i
              \,+\,a2\Delta f'_i
              \,+\,a\frac{4\Delta^2}{2}f''_i
              \,+\,a\frac{8\Delta^3}{6}f^3_i
              \,+\,a\frac{16\Delta^4}{24}f^4_i
              \,+\,a\frac{32\Delta^5}{125}f^5_i  
              \,+\,\ldots  
              \notag\\
    bf_{i+1}&\,=\,bf_i
              \,+\,b\Delta f'_i
              \,+\,b\frac{\Delta^2}{2}f''_i
              \,+\,b\frac{\Delta^3}{6}f^3_i
              \,+\,b\frac{\Delta^4}{24}f^4_i
              \,+\,b\frac{\Delta^5}{125}f^5_i  
              \,+\,\ldots  
              \notag\\
    cf_{i-1}&\,=\,cf_i
               \,-\,c\Delta f'_i
               \,+\,c\frac{\Delta^2}{2}f''_i
               \,-\,c\frac{\Delta^3}{6}f^3_i
               \,+\,c\frac{\Delta^4}{24}f^4_i
               \,-\,c\frac{\Delta^5}{125}f^5_i  
               \,+\,\ldots  
               \notag\\
    df_{i-2}&\,=\,df_i
             \,-\,d2\Delta f'_i
             \,+\,d\frac{4\Delta^2}{2}f''_i
             \,-\,d\frac{8\Delta^3}{6}f^3_i
             \,+\,d\frac{16\Delta^4}{24}f^4_i
             \,-\,d\frac{32\Delta^5}{125}f^5_i  
             \,+\,\ldots
             \notag
  \end{align}
}
tales que
{%\small
  \begin{align}
    a
    \,+\,b
    \,+\,c
    \,+\,d
    &\,=\,\alpha \notag\\
    2a
    \,+\, b
    \,-\, c
    \,-\,2d
    &\,=\,\beta \notag\\
    a\frac{4}{2}
    \,+\,b\frac{1}{2}
    \,+\,c\frac{1}{2}
    \,+\,d\frac{4}{2}
                         &\,=\,0 \notag\\
    a\frac{8}{6}
    \,+\,b\frac{1}{6}
    \,-\,c\frac{1}{6}
    \,-\,d\frac{8}{6}
                         &\,=\,0 \notag\\  
    a\frac{16}{24}
    \,+\,b\frac{1}{24}
    \,+\,c\frac{1}{24}
    \,+\,d\frac{16}{24}
                         &\,=\,0 \notag
%    a\frac{32\Delta^5}{125}f^5_i  
%    \,+\,b\frac{\Delta^5}{125}f^5_i  
%    \,-\,c\frac{\Delta^5}{125}f^5_i  
%    \,-\,d\frac{32\Delta^5}{125}f^5_i  
%                         &\,=\,0 \notag 
  \end{align}
}

\end{document}
